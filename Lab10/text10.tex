\documentclass[12pt, legalpaper]{article}
\usepackage[utf8]{inputenc}
\usepackage[english, russian]{babel}
\usepackage[T2A]{fontenc}
\usepackage{amssymb}
\usepackage{amsmath}

\begin{document}

2) множество номеров $n > n_{\varepsilon}$, т.ч. $|a_n - a| \ge \varepsilon$ конечно или пусто.
Пусть $n'_{\varepsilon}$ --- наибольший из этих номеров, если такие номера имеются, 
и $n'_{\varepsilon} = n_{\varepsilon}$ в противном случае. 
Тогда $\forall n > n'_{\varepsilon}:|a_n - a| < \varepsilon$, т.е. с увеличением номеров $n$ члены 
последовательности ($a_n$) могут только приблизиться к $a$.

Итак, если для последовательности ($a_n$) для некоторого $\varepsilon_0 > 0$ выполняется пункт 1), 
то естественно считать, что ($a_n$) не стремится к $a$ при возрастании $n$. 
Если же $\forall\varepsilon > 0$ 
наблюдается ситуация, описываемая в пункте 2), то это вполне согласуется с представлениями 
о стремлении $a_n$ к $a$. Таким образом, мы приходим к следующему определению.

\textbf{Определение 2.} \textit{Последовательность} ($a_n$) \textit{сходится (стремится) к числу} $a$ \textit{при} $n \to +\infty$,
\textit{если} $\forall\varepsilon > 0\ \exists n_{\varepsilon} \in \mathbb N$, \textit{т.ч.} $\forall n > n_{\varepsilon}:|a_n - a| < \varepsilon$.
\textit{При этом пишут} $\lim_{n\to+\infty} a_n = a$\textit{, или} $a_n \to a$ \textit{при} $n \to +\infty$.

Переформулируем определение предела в терминах окрестностей:

\begin{center}
	$$
	\lim_{n\to+\infty} a_n=a \stackrel{def}{\Leftrightarrow} \forall\varepsilon>0\ \exists n_{\varepsilon} \in \mathbb N\mbox{, т.ч. }\forall n > n_{\varepsilon}:a_n \in U(a, \varepsilon).
	$$
\end{center}

Значит, последовательность ($a_n$) сходится к $a$ тогда и только тогда, когда в любой заданный интервал 
($a - \varepsilon$, $a + \varepsilon$) попадают все члены последовательности, 
начиная с некоторого номера $n_{\varepsilon}+$1. А тогда вне 
любой окрестности $U(a, \varepsilon)$ может находиться лишь конечное число 
членов последовательности. Верно и обратное, что если вне произвольной 
$\varepsilon$-окрестности числа $a$ имеется лишь конечное число членов 
последовательности (это число зависит от рассматриваемой окрестности), 
то $a$ является пределом последовательности.

\textbf{Пример.} Пусть $a_n = (-1)^{n}$. Для $\varepsilon < 2$ в окрестностях 
$U(\pm1,\varepsilon)$ содержится бесконечно много членов последовательности, 
но и вне их также находится бесконечно много членов последовательности. 
Следовательно, $-1$ и 1 не могут быть пределами рассматриваемой последовательности. Далее, 
у любого числа $a \ne \pm1$ легко построить окрестность, не содержащую ни одного члена данной 
последовательности: достаточно взять $\varepsilon \doteq \min(|a - 1|, |a + 1|)$. 
Значит, никакое действительное число не является пределом исследуемой последовательности.

Дадим определение стремления последовательности к $-\infty$, $+\infty$ и $\infty$. 
Для этого в окрестном определении предела надо заменить



\begin{center}
	37
\end{center}
\end{document}
